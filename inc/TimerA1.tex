% !TEX root = ../main.tex
\section{实验\chinese{section}}
\subsection{实验题目}
Timer\_A实验
\subsection{实验目的}
定时模式-Timer\_A0,增/减计数模式,时钟源SMCLK,TA0CCR0 = 50000,ISR内翻转P4.1。
\par\indent TACLK = SMCLK = default DCOCLKDIV。
\par\indent 示波器观察P4.1输出波形,与实验5.1对比,可得出什么结论。
\subsection{实验仪器和设备}
计算机、开发板、示波器、信号源、电源、Code Composer Studio v5、串口调试助手等。
\subsection{实验步骤}
\begin{lstlisting}[language=C]
/****************************************
--|RST P4.1|--> LED_Yellow
****************************************/
\end{lstlisting}
\par\indent 关闭看门狗,开总中断,设置Timer\_A模块为增减计数模式,并配置其他功能寄存器。进入LPM3低功耗模式,当Timer\_A发生中断时,查询TAIV内的标志位,确定中断源并进行处理。
\subsection{程序清单}
\lstinputlisting{src/code/TimerA1.c}
\clearpage
\subsection{实验结果记录与分析}
\begin{figure}[htbp]
	\centering
	\caption{TimerA}
	\label{TimerA1}
	\includegraphics[width=8cm]{bitmap/bmp/TimerA1.bmp}
\end{figure}
\par\indent 增减计数模式和增计数模式的区别主要是在相同的寄存器取值下,增减计数模式是增计数模式周期的2倍,相应的,如果有PWM波输出的话,占空比也变成了\((0.5-0.5\times\)增计数模式的占空比\()\)。
\subsection{遇到的问题与解决方法}
\begin{enumerate}
	\item 在使用示波器上犯了错。一开始误以为探针是参考,探针上的夹子是信号,但实际上是反过来的。把之前用过示波器的实验的图又重新截取了正确的图代替。
\end{enumerate}

